\documentclass{article}
\usepackage{MazzoleniNotation}
\usepackage{geometry}
\geometry{
	letterpaper,
	left=1in,
	right=1in,
	top=1in,
	bottom=1in,
}

\begin{document}
	
	\title{Mazzoleni Notation Development Document}
	\author{Engineering Mechanics and Space Systems Laboratory\\
	Dr. Andre Mazzoleni, Director
	Rodney Metoyer\\
	Chris Yoder}
	
	\maketitle
	
	\section{Overview}
	The purpose of this document is to facilitate communication while developing a standard style and syntax for Mazzoleni notation in Latex.
	
	\section{Style Guide}
	The following guidelines should be followed:
	\begin{itemize}
		\item Commands should start with a lowercase letter
		\item Use camel case for commands
		\item Use snake case for labels
		\item Arguments should in order by precedence
			\begin{itemize}
				\item Frame
				\item Vector
				\item Point
			\end{itemize}
	\end{itemize}
		
	% Use the commands
	\section{Frames}
	A frame (sometimes \textit{reference frame}) defines a space. It may be thought of as a massless rigid body, and rigid bodies often serve as fixtures for frames. As a conceptual entity, a frame has meaning independent of basis vectors, rigid bodies, or coordinate systems, but the easiest way to define a frame is with a set of basis vectors. In the Mazzoleni notation, basis vectors are expressed with they frame that they are referring to explicitly noted, as shown in \ref{eqn_basis}.
	\begin{equation}
	\label{eqn_basis}
	\basisVec{i}{B}
	\end{equation}
	
	\begin{equation}
	\label{eqn_frame}
	\refFrame{B}
	\end{equation}
	We define a frame by specifying the point at which the origin is located and a basis set, as in \ref{eqn_framedef}.
	\begin{equation}
	\label{eqn_framedef}
	\refFrame{A} = \refFrameDefIJK{A}{A}
	\end{equation}
	
	\section{Position Vectors}
	Position vectors may be independent (i.e. have only direction and magnitude) or defined with respect to two points in space. In both cases, position vectors may be expressed in a frame (i.e. with respect to a vector basis) or independent of a basis. The independent form is given by \ref{eqn1}.
	\begin{equation}
		\label{eqn1}
		\gVec{r}
	\end{equation}
	The default point defined form is given by \ref{eqn2}.
	\begin{equation}
		\label{eqn2}
		\posVecP{r}{a}{b}
	\end{equation}
	A vector is typically parameterized when it is expressed in a frame, although it does not need to be. Position vectors are often parameterized with x, y, and z as the scalar values when projected on the \gVec{i}, \gVec{j}, and \gVec{k} basis vectors, respectively. In that case, the position vector expressed in a frame may be written as \ref{eqn_exp}.
	\begin{equation}
		\label{eqn_exp}
		\gVec{r} = \gVecExp{O}{x}{y}{z}
	\end{equation} 
	Or, using i-j-k notation, the position vector may be expressed in a frame as \ref{eqn_exp_ijk}. This is the standard way to express a vector function in Mazzoleni notation.
	\begin{equation}
	\label{eqn_exp_ijk}
	\posVecP{r}{a}{b} = \gVecExpIJK{O}{x}{y}{z}
	\end{equation}
	
	
	\section{Velocity Vectors}
	Position vectors are differentiated in a frame to produce a velocity vector as in \ref{eqn_vec}. 
	\begin{equation}
	\label{eqn_vec}
	\velVec{A}{v}
	\end{equation}
	Velocity vectors can also be defined with respect to points as in \ref{eqn_vecP}.
	\begin{equation}
	\label{eqn_vecP}
	\velVecP{A}{v}{a}{b}
	\end{equation}
	
	Angular velocity is defined between two frames. Thus, \ref{eqn_angVel} is the vector describing the angular velocity of frame \refFrame{B} with respect to frame \refFrame{A}.
	\begin{equation}
	\label{eqn_angVel}
	\angVel{A}{B}
	\end{equation}
	
	\section{Angular Momentum}
	The definition of angular momentum is given by \ref{eqn_angMomFull}
	\begin{equation}
	\label{eqn_angMomFull}
	\angMomFull{A}{h}{c}{a}{b} = \posVecP{r}{a}{b} \times m\velVecP{A}{v}{a}{c}
	\end{equation}
	This is the form that will be used more often.
	\begin{equation}
	\label{eqn_angMom}
	\angMom{O}{h}{a}{b} = \posVecP{r}{a}{b} \times m\velVecP{A}{v}{a}{O}
	\end{equation}
	Sometimes for a system of particles it is convenient to shorten the notation to a point and a frame. In that case use command that makes \ref{eqn_angMomS}.
	\begin{equation}
	\label{eqn_angMomS}
	\angMomS{O}{h}{B,sys} = \posVecP{r}{a}{b} \times m\velVecP{A}{v}{a}{O}
	\end{equation}
	
	\section{Command Tables}
	This section provides a table of the commands.
	
	% Frame commands
	\begin{table}[h!]
		\caption{Reference frame commands.}
	\begin{tabular}[]{ p{1.75in} | p{1.25in} | p{3in} }	
		\hline	
		\textbf{Command} & \textbf{Renders} & \textbf{Description} \\ 
		\hline
		\verb|\refFrame{O}| & \refFrame{O} & Reference frame \\
		\verb|\refFrameDefIJK{A}{A}| & \refFrameDefIJK{A}{A} & Frame definition \\
		\hline		
	\end{tabular}
	\end{table}
	
	% Generic vector commands
	\begin{table}[h!]
		\caption{Generic vector commands.}
		\begin{tabular}[]{ p{1.75in} | p{1.25in} | p{3in} }	
			\hline	
			\textbf{Command} & \textbf{Renders} & \textbf{Description} \\ 
			\hline
			\verb|\gVec{q}| & \gVec{q} & Generic vector \\  
			\verb|\basisVec{i}{B}| & \basisVec{i}{B} & Basis vector \\
			\verb|\gVecExp{O}{x}{y}{z}| & \gVecExp{O}{x}{y}{z} & Generic vector expressed in a frame \\
			\verb|\gVecExpIJK{O}{x}{y}{z}| & \gVecExpIJK{O}{x}{y}{z} & Generic vector expressed in a frame \\
			\hline		
		\end{tabular}
	\end{table}
	
	% Position vector commands
	\begin{table}[h!]
		\caption{Position vector commands.}
		\begin{tabular}[]{ p{1.75in} | p{1.25in} | p{3in} }	
			\hline	
			\textbf{Command} & \textbf{Renders} & \textbf{Description} \\ 
			\hline
			\verb|\posVec{r}| & \posVec{r} & Basis independent position vector \\  
			\verb|\posVecP{r}{a}{b}| & \posVecP{r}{a}{b} & Point dependent position vector \\
			\hline		
		\end{tabular}
	\end{table}
	
	% Frame dependent vector commands
	\begin{table}[h!]
		\caption{Frame dependent vector commands.}
		\begin{tabular}[]{ p{2in} | p{1.0in} | p{3in} }	
			\hline	
			\textbf{Command} & \textbf{Renders} & \textbf{Description} \\ 
			\hline
			\verb|\velVec{O}{v}| & \velVec{O}{v} & Velocity vector \\
			\verb|\velVecP{A}{v}{a}{b}| & \velVecP{A}{v}{a}{b} & Velocity vector with points noted \\
			\verb|\angMomFull{A}{h}{c}{a}{b}| & \angMomFull{A}{h}{c}{a}{b} & Fully defined angular momentum vector \\
			\verb|\angMom{O}{h}{a}{b}| & \angMom{O}{h}{a}{b} & Abridged angular momentum vector \\
			\verb|\angMomS{O}{h}{B,sys}| & \angMomS{O}{h}{B,sys} & Further abridged angular momentum vector \\
			\hline		
		\end{tabular}
	\end{table}
	
	
\end{document}
	
	